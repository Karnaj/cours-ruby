\addchap{Introduction}
Vous rêvez d’apprendre à programmer, mais vous ne savez pas quel langage apprendre ? Pourquoi ne pas apprendre le Ruby ? Ruby est un langage de programmation \textbf{libre} et \textbf{dynamique},qui met l’accent sur la \textbf{simplicité et la productivité}. Sa \textbf{syntaxe} élégante en facilite la lecture et l’écriture.

\begin{Information}
\begin{description}
	\item[Prérequis] : connaître les quatre opérations mathématiques élémentaires.
	\item [Prérequis optionnels] : 
	  \begin{itemize} 
      \item avoir déjà lu ce \href{https://zestedesavoir.com/tutoriels/531/les-bases-de-la-programmation}{tutoriel} qui introduit la programmation et aide à choisir un premier langage ;   
		  \item connaître la notion de nombres relatifs ;
		  \item avoir des bases dans l’utilisation de la ligne de commande (savoir l’ouvrir et savoir entrer une commande).
	  \end{itemize}				
	\item[Objectifs] : apprendre les bases de la programmation en Ruby. 
\end{description}
\end{Information}

L’apprentissage de la programmation peut-être une course pour certains.Mais cette idée n’est bonne que s’il s’agit d’une course d’\textbf{endurance}. En effet, le but n’est pas de finir le tutoriel le plus rapidement possible. Le but est de prendre son temps pour comprendre toutes les notions et surtout pour les \textbf{appliquer}. Seule la pratique permet de progresser, et il vaut mieux rester une semaine sur un chapitre pour maîtriser toutes les notions qui y sont traitées, plutôt que de passer une semaine sur tout le tutoriel.

\subsection*{Remerciements}

Nous voulons remercier plusieurs personnes avant de commencer :

\begin{itemize}
\item \href{https://zestedesavoir.com/membres/voir/baptisteguil/}{baptisteguil} le créateur originel et premier rédacteur de ce tutoriel ;
\item \href{https://zestedesavoir.com/membres/voir/Dominus Carnufex}{Dominus Carnufex} pour son travail monstrueux et très rapide de corrections orthotypographiques ;
\item \href{https://zestedesavoir.com/membres/voir/Arius/}{Arius} pour son formidable travail de validation ;
\item Tous les membres qui ont apporté leurs conseils, leurs avis et leurs corrections.
\end{itemize}
